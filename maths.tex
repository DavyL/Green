\documentclass[12pt]{article}

\title{Fonction de Green}
\author{DAVY Leo, KAOUAH Mohammed, ABRIBAT Clement}

\begin{document}
\maketitle
\section{Introduction à l'equation de Dyson}
La fonction de Green peut etre ecrite a l'aide de l'equation de Dyson. 
\begin{equation}
	G = G_0 + G_0 \Sigma[G] G
\end{equation}
Avec $G$ la fonction de Green qui représente le système, $G_0$ le systeme a l'etat initial qui est mesure, et $\Sigma$ une fonctionnelle qui correspond a l'energie propre du systeme.
$\Sigma$ s'ecrit :
\begin{equation}
	\Sigma[G] = \Sigma_{Hartree}[G] + \Sigma_{XC}[G]
\end{equation}
\begin{equation}
	\Sigma_{Hartree}[G] = \int \rho(r) \frac{1}{\mid r - r'\mid }\mathrm{d}r' 
\end{equation}
avec $v_c$ qui correspond a l'interaction Coulombienne, $\rho(r)$ correspond a la densite electronique, et r et r' qui correspondent aux positions des particules en interaction.
\begin{equation}
	v_c =  \frac{1}{\mid r - r'\mid }
\end{equation}
\begin{equation}
	\Sigma_{XC} = G \Gamma[G] v_c
\end{equation}
\begin{equation}
	\Gamma[G] = 1 + G^2 \frac{\mathrm{d} \Sigma_{XC}[G]}{\mathrm{d}G} \Gamma[G]
\end{equation}
Cependant ces equations dependant de fonctionnelles de G elles sont trop compliquees a resoudre. Nous allons donc etudier le systeme en dimension spatio-temporelle nulle et renommer les variables comme suit pour plus de clarte.
\begin{equation}
	\Sigma \longrightarrow S
\end{equation}
\begin{equation}
	G \longrightarrow y
\end{equation}
\begin{equation}
	\Gamma \longrightarrow g
\end{equation}
\begin{equation}
	v_c \longrightarrow u
\end{equation}
\section{Dyson}

\begin{equation}
	y = y_0 + y_0 S(y) y
\end{equation}
Et la solution 
\begin{equation}
	Y = \frac{1}{V} \sqrt{1+2V} 
\end{equation}
avec
\begin{equation}
	V = uy_0^2
\end{equation}
\begin{equation}
\label{sigma}
	S(y) = S_{XC}(y) + S_{Hartree}(y)
\end{equation}
\begin{equation}
\label{Hartree} 
	S_{Hartree}(y) = -uy
\end{equation}
On étudie dans un premier temps $S_{XC}(y)$ 
\begin{equation}
\label{XC} 
	S_{XC}(y) = \frac{1}{2} u y g(y)
\end{equation}
\begin{equation} 
\label{gamma} 
	g(y) = 1 + y^2 \frac{dS_{XC}(y)}{dy} g(y_0)
\end{equation}
On combine les equations \ref{XC} et \ref{gamma} : 
\begin{equation} 
\label{XC_dev}
	S_{XC}(y) = \frac{1}{2} u y (1 + y^2 \frac{dS_{XC}(y)}{dy} g(y_0))
\end{equation}
\begin{equation} 
\label{XC_prime} 
	\frac{dS_{XC}(y)}{dy} = \frac{d(\frac{1}{2} u y g(y_0))}{y} = \frac{1}{2} u g(y_0)
\end{equation}
Or $g(y_0) = 1$ ,donc on combine maintenant les equations \ref{XC_dev}  et \ref{XC_prime}.
\begin{equation}
\label{XC_final} 
	S_{XC}(y) = \frac{1}{2} u y + \frac{1}{4} u^2 y^3 
\end{equation}
On obtient donc avec \ref{sigma}: 
\begin{equation}
\label{Sigma_final} 
	S(y) = -uy +  \frac{1}{2} u y + \frac{1}{4} u^2 y^3 
\end{equation}
Finalement, on obtient l'equation suivante que l'on va chercher à simplifier afin de la resoudre.
\begin{equation}
	y = -\frac{1}{2} u y_0 y^2 + \frac{1}{4} u^2 y_0 y^4 + y_0
\end{equation}
\begin{equation}
	\frac{1}{4} u^2 y_0 y^4 -\frac{1}{2} u y_0 y^2 - y + y_0 = 0
\end{equation}
On multiplie l'equation précedente par $$ \frac{4}{u^2 y_0}$$
On obtient : 
\begin{equation}
	y^4 -\frac{2}{u} y^2 - \frac{4y}{y_0  u^2} + \frac{4}{u^2} = 0
\end{equation}
Posons $$ U = \frac{2}{u}$$
\begin{equation}
\label{Dyson_final} 
	y^4 - U y^2 - U^2\frac{y}{y_0} + U^2 = 0
\end{equation}
Il faut donc maintenant resoudre \ref{Dyson_final}
%
Afin de résoudre cette équation de degré 4 nous allons utiliser la m\'ethode de Lagrange.
\section{R\'esolution de l'\'equation de Dyson}
Donc \ref{Dyson_final} est de la forme :
\begin{equation}
y^4 + a y^2 + b y + c 
\end{equation}
Avec $$a = -U$$ et $$b = -\frac{U^2}{y_0}$$ et $$c = u^2$$
et on notera $\alpha, \beta, \gamma et \delta$ ses solutions. 
On pose d, e et f les solutions de l'\'equation suivante : 
\begin{equation}
y^3 - a y^2 -4cy - b^2 + 4 ac = 0
\end{equation}
On cherche donc les solutions d'une equation du troisi\'eme degr\'e.
On pose $$y = z - U/3$$
On obtient donc l'\'equation :
\begin{equation}
(z - \frac{U}{3})^3  + U (z - \frac{U}{3})^2 - 4 U ^2(z - \frac{U}{3}) + \frac{U^2}{y_0} - 4U^3 = 0
\end{equation}
\begin{equation}
z^3-z^2U+zU^2 - \frac{U^3}{27}+Uz^2-2\frac{zU^2}{3}+ \frac{U^3}{9} - 4U^2z+\frac{4}{3}U^3+\frac{U^2}{y_0}-4U^3 = 0
\end{equation}

\begin{equation}
\label{deg3}
z^3 - z\frac{11}{3}U^2 - u^3\frac{70}{27}+\frac{U^2}{y_0} = 0 
\end{equation}
On peut \'ecrire \ref{deg3} sous la forme :
\begin{equation}
z^3+pz+q = 0
\end{equation}
Avec $$p = -\frac{11}{3}U^2$$ et $$q = - u^3\frac{70}{27}+\frac{U^2}{y_0} $$
On peut donc poser :
\begin{equation}
z^2+27qz - 27p^3 = 0 
\end{equation}
avec $z_{+}$ et $z_{-}$ les solutions de cette \'equation.
On trouve donc $$\Delta = (27q)^2 + 4*27p^3$$
\begin{equation}
\Delta = \frac{U^5*70^2*y_0^2+ 140*27U^5+27^2U^4}{y_0^2} - 4*11^3U^6 
\end{equation}
On obtient donc :
\begin{equation}
z_+ = \frac{U^3 70 + \frac{27U^2}{y_0} +\sqrt{\Delta}}{2} 
\end{equation}
\begin{equation}
z_- = \frac{U^3 70 + \frac{27U^2}{y_0} -\sqrt{\Delta}}{2}  
\end{equation}
On a aussi
\begin{equation}
 \sqrt{\Delta} = \sqrt{\frac{U^5*70^2*y_0^2+ 140*27U^5+27^2U^4}{y_0^2} - 4*11^3U^6 }
\end{equation}

\begin{equation}
 \sqrt{\Delta} = \frac{U^2}{y_0} \sqrt{U*70^2*y_0^2+ 140*27U+27^2 -4*11^3*U^2*y_0^2} = \frac{U^2}{y_0} \sqrt{A}
\end{equation}

On pose maintenant $k$ (respectivement $l$ ) comme \'etant la racine cubique de $z_+$ ( respectivement $z_-$ ).
 Avec $j = e^{i\frac{2\pi}{3}}$
\begin{equation}
\left\{ \begin{array}{rl}
d = \frac{1}{3} ( k + l + U)\\
e = \frac{1}{3} (kj^2 + lj + U)\\
f = \frac{1}{3} (kj + lj^2 + U)
\end{array} \right.
\end{equation} 


\begin{equation}
\left\{ \begin{array}{rl}
d = \frac{U}{3} ( \sqrt[3]{35+\frac{u}{2y_0}(27 + \sqrt{A})}+\sqrt[3]{35+\frac{u}{2y_0}(27 - \sqrt{A})} + 1)\\
e = \frac{U}{3} (\sqrt[3]{35+\frac{u}{2y_0}(27 + \sqrt{A})} e^{i\frac{4\pi}{3}}  +   \sqrt[3]{35+\frac{u}{2y_0}(27 - \sqrt{A})}e^{i \frac{2\pi}{3}} 
+ e^{i\frac{4\pi}{3}} + e^{i\frac{2\pi}{3}} )\\
f = \frac{U}{3} (\sqrt[3]{35+\frac{u}{2y_0}(27 + \sqrt{A})} e^{i\frac{2\pi}{3}}  +   \sqrt[3]{35+\frac{u}{2y_0}(27 - \sqrt{A})}e^{i \frac{4\pi}{3}} 
+ e^{i\frac{2\pi}{3}} + e^{i\frac{4\pi}{3}} )\\
\end{array} \right.
\end{equation}


\begin{equation}
\left\{ \begin{array}{rl}
d = \frac{U}{3} ( \sqrt[3]{35+\frac{u}{2y_0}(27 + \sqrt{A})}+\sqrt[3]{35+\frac{u}{2y_0}(27 - \sqrt{A})} + 1)\\
e = \frac{U}{3} (((35+\frac{u}{2y_0}(27 + \sqrt{A})) e^{i4\pi})^{\frac{1}{3}} + ((35+\frac{u}{2y_0}(27 - \sqrt{A})))e^{i 2\pi})^{\frac{1}{3}}
+ e^{i\frac{4\pi}{3}} + e^{i\frac{2\pi}{3}} )\\
f = \frac{U}{3} (((35+\frac{u}{2y_0}(27 + \sqrt{A})) e^{i2\pi})^{\frac{1}{3}} + ((35+\frac{u}{2y_0}(27 - \sqrt{A}))e^{i 4\pi})^{\frac{1}{3}} 
+ e^{i\frac{2\pi}{3}} + e^{i\frac{4\pi}{3}} )\\
\end{array} \right.
\end{equation}
Or
\begin{equation}
 e^{i4\pi} = e^{i2\pi} = 1
\end{equation}
Et
\begin{equation}
 j^2+j= e^{i\frac{4\pi}{3}} + e^{i\frac{2\pi}{3}} = -1
\end{equation}

\begin{equation}
\left\{ \begin{array}{rl}
d = \frac{U}{3} ( \sqrt[3]{35+\frac{u}{2y_0}(27 + \sqrt{A})}+\sqrt[3]{35+\frac{u}{2y_0}(27 - \sqrt{A})} + 1)\\
e = f = \frac{U}{3} (\sqrt[3]{(35+\frac{u}{2y_0}(27 + \sqrt{A}))} + \sqrt[3]{(35+\frac{u}{2y_0}(27 - \sqrt{A}))} -1 )\\
\end{array} \right.
\end{equation}
Par ailleurs, on a :
\begin{equation}
\left\{ \begin{array}{rl}
\alpha + \beta + \gamma + \delta = 0 \\
\alpha \beta + \gamma \delta = d \\
\alpha \gamma + \beta \delta = e \\
\alpha \delta + + \beta \gamma = f \\
\end{array} \right.
\Longleftrightarrow
\left\{ \begin{array}{rl}
\alpha + \beta + \gamma + \delta = 0 \\
(\alpha + \beta )( \gamma + \delta ) = e+f \\
(\alpha + \gamma)( \beta + \delta) = d +f \\
(\alpha  + \delta)( \beta + \gamma) = d+e \\
\end{array} \right.
\end{equation}
On d\'eduit ensuite des \'equations de ce syst\`eme :
\newline
$\alpha + \beta = \rho_1$ , $\gamma + \delta = -\rho_1 $ avec $\rho_1 = \sqrt{-e-f} $
\newline
$\alpha + \delta = \rho_2$ , $\beta + \gamma = -\rho_2$ avec $\rho_2 = \sqrt{-d -e}$
\newline
$\alpha + \gamma = \rho_3$ , $\beta + \delta = -\rho_3$ avec $\rho_3 = \sqrt{-d-f}$ 
On a donc : 
\begin{equation}
 \rho_1 = \sqrt{-e-f} = \sqrt{-2e} = \sqrt{\frac{-2U}{3} (\sqrt[3]{(35+\frac{u}{2y_0}(27 + \sqrt{A}))} + \sqrt[3]{(35+\frac{u}{2y_0}(27 - \sqrt{A}))} -1 )}
\end{equation}
\begin{equation}
 \rho_2 = \sqrt{-d-e} = \sqrt{\frac{-2U}{3} (\sqrt[3]{(35+\frac{u}{2y_0}(27 + \sqrt{A}))} + \sqrt[3]{(35+\frac{u}{2y_0}(27 - \sqrt{A})})}
\end{equation}
\begin{equation}
 \rho_3 = \sqrt{-d-f} = \sqrt{-d-e} = \rho_2
\end{equation}
\begin{equation}
 \rho_1 = \sqrt{-\frac{2U}{3}(j^2+j}) = \sqrt{\frac{2U}{3}}
\end{equation}
\begin{equation}
 \delta = \frac{1}{2}(-\rho_1) = -\sqrt{\frac{2U}{3}} 
\end{equation}
$\delta$ est donc une des solutions de l'\'equation de Dyson.



\end{document}    
