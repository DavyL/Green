\documentclass[12pt]{article}

\title{Fonction de Green}
\author{DAVY Leo, KAOUAH Mohammed, ABRIBAT Clement}

\begin{document}
\maketitle

\section{Dyson}

\begin{equation}
 y = y_0 + y_0 S(y) y
\end{equation}
Et la solution 
\begin{equation}
 Y = \frac{1}{V} \sqrt{1+2V} 
\end{equation}
avec
\begin{equation}
 V = uy_0^2
\end{equation}
\begin{equation}
\label{sigma}
 S(y) = S_{XC}(y) + S_{Hartree}(y)
\end{equation}
\begin{equation}
\label{Hartree} 
 S_{Hartree}(y) = -uy
\end{equation}
On étudie dans un premier temps $S_{XC}(y)$ 
\begin{equation}
\label{XC} 
 S_{XC}(y) = \frac{1}{2} u y g(y)
\end{equation}
\begin{equation} 
\label{gamma} 
 g(y) = 1 + y^2 \frac{dS_{XC}(y)}{dy} g(y_0)
\end{equation}
On combine les equations \ref{XC} et \ref{gamma} : 
\begin{equation} 
\label{XC_dev}
 S_{XC}(y) = \frac{1}{2} u y (1 + y^2 \frac{dS_{XC}(y)}{dy} g(y_0))
\end{equation}
\begin{equation} 
\label{XC_prime} 
 \frac{dS_{XC}(y)}{dy} = \frac{d(\frac{1}{2} u y g(y_0))}{y} = \frac{1}{2} u y g(y_0)
\end{equation}
Or $g(y_0) = 1$ ,donc on combine maintenant les equations \ref{XC_dev}  et \ref{XC_prime}.
\begin{equation}
\label{XC_final} 
 S_{XC}(y) = \frac{1}{2} u y + \frac{1}{4} u^2 y^3 
\end{equation}
On obtient donc avec \ref{sigma}: 
\begin{equation}
\label{Sigma_final} 
 S(y) = -uy +  \frac{1}{2} u y + \frac{1}{4} u^2 y^3 
\end{equation}
Finalement, on obtient l'équation suivante que l'on va chercher à simplifier afin de la resoudre.
\begin{equation}
 y = -\frac{1}{2} u y_0 y^2 + \frac{1}{4} u^2 y_0 y^4 + y_0
\end{equation}
\begin{equation}
 \frac{1}{4} u^2 y_0 y^4 -\frac{1}{2} u y_0 y^2 - y + y_0 = 0
\end{equation}
On multiplie l'équation précedente par $$4 \frac{1}{u^2 y_0}$$
On obtient : 
\begin{equation}
 y^4 -\frac{2}{u} y^2 - \frac{4y}{y_0  u^2} + \frac{4}{u^2} = 0
\end{equation}
Posons $$ U = \frac{2}{u}$$
\begin{equation}
\label{Dyson_final} 
 y^4 - U y^2 - U^2\frac{y}{y_0} + U^2 = 0
\end{equation}
Il faut donc maintenant resoudre \ref{Dyson_final}

%


\end{document}    
