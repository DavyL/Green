\documentclass[12pt]{article}

\title{Fonction de Green}
\author{DAVY Leo, KAOUAH Mohammed, ABRIBAT Clement}

\begin{document}
\maketitle
\section{Introduction à l'equation de Dyson}
La fonction de Green peut etre ecrite a l'aide de l'equation de Dyson. 
\begin{equation}
	G = G_0 + G_0 \Sigma[G] G
\end{equation}
Avec $G$ la fonction de Green qui représente le système, $G_0$ le systeme a l'etat initial qui est mesure, et $\Sigma$ une fonctionnelle qui correspond a l'energie propre du systeme.
$\Sigma$ s'ecrit :
\begin{equation}
	\Sigma[G] = \Sigma_{Hartree}[G] + \Sigma_{XC}[G]
\end{equation}
\begin{equation}
	\Sigma_{Hartree}[G] = \int \rho(r) \frac{1}{\mid r - r'\mid }\mathrm{d}r' 
\end{equation}
avec $v_c$ qui correspond a l'interaction Coulombienne, $\rho(r)$ correspond a la densite electronique, et r et r' qui correspondent aux positions des particules en interaction.
\begin{equation}
	v_c =  \frac{1}{\mid r - r'\mid }
\end{equation}
\begin{equation}
	\Sigma_{XC} = G \Gamma[G] v_c
\end{equation}
\begin{equation}
	\Gamma[G] = 1 + G^2 \frac{\mathrm{d} \Sigma_{XC}[G]}{\mathrm{d}G} \Gamma[G]
\end{equation}
Cependant ces equations dependant de fonctionnelles de G elles sont trop compliquees a resoudre. Nous allons donc etudier le systeme en dimension spatio-temporelle nulle et renommer les variables comme suit pour plus de clarte.
\begin{equation}
	\Sigma \longrightarrow S
\end{equation}
\begin{equation}
	G \longrightarrow y
\end{equation}
\begin{equation}
	\Gamma \longrightarrow g
\end{equation}
\begin{equation}
	v_c \longrightarrow u
\end{equation}
\section{Dyson}

\begin{equation}
	y = y_0 + y_0 S(y) y
\end{equation}
Et la solution 
\begin{equation}
	Y = \frac{1}{V} \sqrt{1+2V} 
\end{equation}
avec
\begin{equation}
	V = uy_0^2
\end{equation}
\begin{equation}
\label{sigma}
	S(y) = S_{XC}(y) + S_{Hartree}(y)
\end{equation}
\begin{equation}
\label{Hartree} 
	S_{Hartree}(y) = -uy
\end{equation}
On étudie dans un premier temps $S_{XC}(y)$ 
\begin{equation}
\label{XC} 
	S_{XC}(y) = \frac{1}{2} u y g(y)
\end{equation}
\begin{equation} 
\label{gamma} 
	g(y) = 1 + y^2 \frac{dS_{XC}(y)}{dy} g(y_0)
\end{equation}
On combine les equations \ref{XC} et \ref{gamma} : 
\begin{equation} 
\label{XC_dev}
	S_{XC}(y) = \frac{1}{2} u y (1 + y^2 \frac{dS_{XC}(y)}{dy} g(y_0))
\end{equation}
\begin{equation} 
\label{XC_prime} 
	\frac{dS_{XC}(y)}{dy} = \frac{d(\frac{1}{2} u y g(y_0))}{y} = \frac{1}{2} u g(y_0)
\end{equation}
Or $g(y_0) = 1$ ,donc on combine maintenant les equations \ref{XC_dev}  et \ref{XC_prime}.
\begin{equation}
\label{XC_final} 
	S_{XC}(y) = \frac{1}{2} u y + \frac{1}{4} u^2 y^3 
\end{equation}
On obtient donc avec \ref{sigma}: 
\begin{equation}
\label{Sigma_final} 
	S(y) = -uy +  \frac{1}{2} u y + \frac{1}{4} u^2 y^3 
\end{equation}
Finalement, on obtient l'equation suivante que l'on va chercher à simplifier afin de la resoudre.
\begin{equation}
	y = -\frac{1}{2} u y_0 y^2 + \frac{1}{4} u^2 y_0 y^4 + y_0
\end{equation}
\begin{equation}
	\frac{1}{4} u^2 y_0 y^4 -\frac{1}{2} u y_0 y^2 - y + y_0 = 0
\end{equation}
On multiplie l'equation précedente par $$ \frac{4}{u^2 y_0}$$
On obtient : 
\begin{equation}
	y^4 -\frac{2}{u} y^2 - \frac{4y}{y_0  u^2} + \frac{4}{u^2} = 0
\end{equation}
Posons $$ U = \frac{2}{u}$$
\begin{equation}
\label{Dyson_final} 
	y^4 - U y^2 - U^2\frac{y}{y_0} + U^2 = 0
\end{equation}
Il faut donc maintenant resoudre \ref{Dyson_final}

%


\end{document}    
